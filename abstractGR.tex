\chapter*{\centering \textgreek{\emph{Περίληψη}}}
%\addtocontents{toc}{\protect\contentsline{chapter}{\protect\numberline{}\textgreek{Περίληψη (Ελληνικά)}}{}{chapter*.%\thepage}}
%\addcontentsline{toc}{chapter}{Abstract (Greek)}
\textgreek{Η παρούσα διπλωματική εξετάζει το πρόβλημα της αναγνώρισης αντικειμένων από εικόνες, των οποίων τα εικονοστοιχεία είναι ταξινομημένα σε μια από 19 κατηγορίες. Στην εργασία χρησιμοποιείται η βάση δεδομένων} Cityscapes\textgreek{ που αποτελείται από 19 διαφορετικές κατηγορίες αντικειμένων η οποία έχει δημιουργηθεί με χρήση κάμερας τοποθετημένη στο εμπρόσθιο μέρος αυτοκινήτου. Οι εικόνες έχουν απαθανατιστεί από 50 διαφορετικές πόλεις της Ευρώπης σε διάφορες εποχές και καιρικές συνθήκες. \par 

Με την χρήση πληροφορίας από έγχρωμες εικόνες κατασκευάζουμε έναν ταξινομητή ο οποίος μπορεί να αναγνωρίσει την κατηγορία αντικειμένων που ανήκει το κάθε εικονοστοιχείο στην εικόνα ως συνάρτηση των τιμών των εικονοστοιχείων αλλά και της δομής που απεικονίζουν. Για την ταξινόμηση χρησιμοποιήσαμε 2 πανομοιότυπες αρχιτεκτονικές πλήρως συνελικτικών νευρωνικών δικτύων (}FCNNs\textgreek{) και 2 διαφορετικές μονάδες μετα-επεξεργασίας.} \par

\textgreek{Στόχος της εργασίας ήταν η δημιουργία διαφορετικών ταξινομητών καθώς και η σύγκριση μεταξύ των μεθόδων, αλλά και η δημιουργία λογισμικού για την οπτικοποίηση των αποτελεσμάτων. Για την οπτικοποίηση των παραπάνω αποτελεσμάτων υλοποιήθηκε λογισμικό που απεικονίζει τα αποτελέσματα των μεθόδων. Για την κατασκευή των παραπάνω μοντέλων γίνεται χρήση των βιβλιοθηκών } Keras \textgreek{ και} Tensorflow, \textgreek{ ενώ για την υλοποίηση του λογισμικού οπτικοποίησης έγινε η χρήση της βιβλιοθήκης} pyQt.
 

\chapter*{\centering \emph{Abstract}}
%\addtocontents{toc}{\protect\contentsline{chapter}{\protect\numberline{}Abstract (English)}{}{chapter*.\thepage}}

This thesis focuses on the problem of recognizing objects from images which are pixel-wise classified in one of 19 various classes. The Cityscapes database introduced in CVPR 2016, consists of 19 various classes of objects created using a camera mounted on automobiles. Images have been recorded in 50 European cities in different seasons and weather conditions.
\par 
Using information from coloured images, a classifier was implemented to recognize the category of objects where each individual pixel belongs to. As part of classification, two different Fully Convolutional Neural Networks models along with another two post processing units were implemented. 
\par 

The aim of this thesis is to create and compare the results from various model architectures, and to also integrate a sophisticated visualizer which presents their results. The tools used in this project are Keras and Tensorflow, as well as pyQt for the implementation of the visualizer.

\chapter*{\centering\emph{\textgreek{Ευχαριστίες}}}
\textgreek{Θα ήθελα να ευχαριστήσω τους επιβλέποντες καθηγητές μου, καθηγητές Γεράσιμο Ποταμιάνο και Αντώνιο Αργυρίου για την υποστήριξη αλλά και την απαραίτητη γνώση και τα κίνητρα που μου έδωσαν μέσα από τα μαθήματα τους ώστε να πραγματοποιηθεί αυτή η διπλωματική. Επίσης, θα ήθελα να ευχαριστήσω τον συνεπιβλέποντα ερευνητή Θεόδωρο Γιαννακόπουλο από το Ε.Κ.Ε.Φ.Ε Δημόκριτος για την υποστήριξη καθώς και τους ανθρώπους από το εργαστήριο Υπολογιστικής Ευφυΐας που μου έδωσαν χώρο και πόρους για να υλοποιηθεί αυτή η διπλωματική.}
