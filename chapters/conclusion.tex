\chapter{\textgreek{Συμπεράσματα και Μελλοντική Εργασία}}

\pagestyle{fancy}
\fancyhf{}
%\fancyhead[OC]{\leftmark}
%\fancyhead[C]{}
%\fancyhead[EC]{\rightmark}
\renewcommand{\footrulewidth}{0.5pt}
\cfoot{\thepage}

\section{\textgreek{Συμπεράσματα}}
\textgreek{Ο σκοπός της συγκεκριμένης διπλωματικής ήταν η εξερεύνηση και η σύγκριση τεχνικών τελευταίας γενιάς στο πεδίο της μηχανικής μάθησης για το πρόβλημα της σημασιολογικής κατάτμησης αντικειμένων από εικόνες αλλά και η δημιουργία κατάλληλου εργαλείου για την προβολή των προβλέψεων από τα μοντέλα. Πιο συγκεκριμένα, επικεντρωθήκαμε στην εφαρμογή μεθόδων βαθιάς μάθησης με χρήση αρχιτεκτονικών ΠΣΝΔ με το μοντέλο γράφων ΤΥΣΠ-ΕΝΔ για την εκτίμηση της κατηγορίας που ανήκει κάθε εικονοστοιχείο της εικόνας. Τα δίκτυα ΠΣΝΔ αποτελούν μια από τις τεχνικές τελευταίας γενιάς στα προβλήματα σημασιολογικής κατάτμησης ειδικά σε συνδυασμό με τους γράφους ΤΥΣΠ καθώς έχουν επιδείξει πολύ καλά αποτελέσματα. 
\par
Μια ενδιαφέρουσα συνεισφορά της εργασίας είναι η χρήση της εκθετικής συνάρτησης ενεργοποίησης η οποία σε συνδυασμό με την σωστή συνάρτηση αρχικοποίησης των βαρών αντιμετωπίζουν το πρόβλημα των νεκρών νευρώνων το οποίο τείνουν να πάσχουν τα βαθειά ΝΔ. Επίσης, αυτή η προσέγγιση είναι πιο αποδοτική καθώς δεν προσθέτει υπολογιστικό κόστος στην εκπαίδευση του ΝΔ σε σχέση με άλλες προσεγγίσεις.
\par
Τέλος, παρουσιάσαμε και συγκρίναμε δύο πανομοιότυπες αρχιτεκτονικές βασισμένες σε ΣΝΔ κωδικοποίησης και αποκωδικοποίησης και πως αυτές ανταποκρίνονται. Είναι φανερό πως το ΣΝΔ με την μονάδα αποκωδικοποίησης με άλμα ολίσθησης αν και διαθέτει μεγαλύτερο αριθμό παραμέτρων, δίνει καλύτερα αποτελέσματα λόγω της μη γραμμικής υπερδειγματοληψίας η οποία διαθέτει παραμέτρους που μαθαίνουν την χαρτογράφηση της υπερδειγματοληψίας κατά την εκπαίδευση.  Επίσης, η χρησιμότητα της συνάρτησης μέσης συχνότητας ισορροπίας, η οποία παίζει σημαντικό ρόλο σε τέτοιου είδος προβλήματα στο στάδιο της εκπαίδευσης, καθώς επιτυγχάνεται μια ισορροπία ως ένα βαθμό μεταξύ της δυσαναλογίας των κλάσεων που υπάρχει στα δεδομένα. 
}
\newpage
\section{\textgreek{Μελλοντική Εργασία}}
\textgreek{Το θέμα της σημασιολογικής κατάτμησης κεντρίζει όλο και περισσότερο το ενδιαφέρον των επιστημόνων καθώς αποτελεί πρόκληση στον κλάδο, ενώ η συνεχής ανάπτυξη της υπολογιστικής δύναμης η οποία είναι απαραίτητη σε συνδυασμό με την δημιουργία καινούριων αυτόνομων μηχανών που παίρνουν αποφάσεις σύμφωνα με τον ακριβή διαχωρισμό των αντικειμένων στο περιβάλλον} \cite{deepNAT, Home3d}, \textgreek{έχουν σαν αποτέλεσμα την μεταστροφή από προβλήματα ανίχνευσης αντικειμένων στην σημασιολογική κατάτμηση.
\par
Στο μέλλον θα θέλαμε να χρησιμοποιήσουμε πιο βαθειά μοντέλα χρησιμοποιώντας λιγότερη υποδειγματοληψία στις εικόνες εισόδου για περισσότερη πληροφορία. Επίσης θα θέλαμε να χρησιμοποιήσουμε περισσότερα δεδομένα, όμως υπάρχει δυσκολία σε αυτό το κομμάτι καθώς θα πρέπει να δημιουργηθούν καινούριες εικόνες με κατηγοριοποιημένα όλα τα εικονοστοιχεία. Μία καλή προσέγγιση θα ήταν η δημιουργία συνθετικών δεδομένων από τις ήδη υπάρχουσες εικόνες. Οι συνθετικές εικόνες δημιουργούνται με εφαρμογή από μια πληθώρα κατάλληλων φίλτρων πάνω στις εικόνες ώστε να δημιουργήσουμε παραλλαγές των εικόνων και ως αποτέλεσμα περισσότερα δεδομένα για την αντιμετώπιση του προβλήματος της υπερμάθησης. Επίσης, θα θέλαμε να δοκιμάσουμε την αρχιτεκτονική με την διγραμμική υπερδειγματοληψία με περισσότερα φίλτρα σε συνδυασμό με την μονάδα ΤΥΣΠ-ΕΝΔ καθώς είναι πιθανόν να υπάρχουν προοπτικές. 
}
\par
\textgreek{Μία διαφορετική κατεύθυνση είναι η χρήση ενός προ-εκπαιδευμένου ΠΣΝΔ το οποίο έχει εκπαιδευτεί σε κάποιο διαφορετικό πρόβλημα. Η χρήση ενός τέτοιου μοντέλου βοηθάει στην εξαγωγή πολύπλοκων χαρακτηριστικών τα οποία μπορούν να τροφοδοτήσουν ένα ΠΣΝΔ όπως το δικό μας. Η εκπαίδευση ενός τέτοιου μοντέλου σε συνδυασμό με το δικό μας μοντέλο θα μπορούσε να προσφέρει καλύτερα αποτελέσματα. Ο μεγαλύτερος αριθμός παρτίδας επίσης, θα μπορούσε να επιφέρει καλύτερα αποτελέσματα, καθώς χρησιμοποιήσαμε μία παρτίδα της τάξης του 4 στην καλύτερη περίπτωση, ο υπολογισμός σε περισσότερα δεδομένα σε κάθε επανάληψη και ανανέωση των κρυμμένων στοιχείων ανά περισσότερα κομμάτια θα μπορούσε να βελτιώσει τα αποτελέσματα. Δυστυχώς, η αύξηση της παρτίδας και ειδικά σε δεδομένα πολύ μεγάλων διαστάσεων απαιτούν και αρκετούς πόρους.
\par 
Τέλος, η παράλληλη μονάδα επεξεργασίας θα μπορούσε να βελτιώσει τα αποτελέσματα αν μπορούσαμε να την χρησιμοποιήσουμε με μεγαλύτερα μεγέθη χαρτών χαρακτηριστικών στην είσοδο, καθώς θα μπορούσαν να αποδώσουν καλύτερα στην αξιοποίηση της πληροφορίας λόγω της μικρότερης συμπίεσης. Επίσης η αύξηση των αριθμών των φίλτρων σε κάθε κλάδο θα μπορούσαν να αποδώσουν θετικά καθώς θα υπήρχαν περισσότεροι χάρτες χαρακτηριστικών, όμως αυτή η επιλογή έρχεται με αντάλλαγμα την αύξηση των παραμέτρων.}
